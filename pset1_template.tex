%
% 6.006 problem set 1 solutions template
%
\documentclass[12pt,twoside]{article}

\usepackage{amsmath}
\usepackage{color}

\input{macros}

\setlength{\oddsidemargin}{0pt}
\setlength{\evensidemargin}{0pt}
\setlength{\textwidth}{6.5in}
\setlength{\topmargin}{0in}
\setlength{\textheight}{8.5in}

\newcommand{\theproblemsetnum}{1}
\newcommand{\releasedate}{Thursday, February 20}
\newcommand{\partaduedate}{Thursday, February 20}
\newcommand{\tabUnit}{3ex}
\newcommand{\tabT}{\hspace*{\tabUnit}}

\title{6.006 PSET 1}

\begin{document}

\handout{Problem Set \theproblemsetnum}{February 6, 2014}

\textbf{All parts are due {\bf \partaduedate} at {\bf 11:59PM}}.
%
Please download the .zip archive for this problem set, and refer to the
\texttt{README.txt} file for instructions on preparing your solutions.
%
Remember, your goal is to communicate. Full credit will be given only
to a correct solution which is described clearly. Convoluted and
obtuse descriptions might receive low marks, even when they are
correct. Also, aim for concise solutions, as it will save you time
spent on write-ups, and also help you conceptualize the key idea of
the problem.

\setlength{\parindent}{0pt}

\medskip

\hrulefill

\medskip

{\bf Your Name:} Eunice Wu

\medskip

{\bf Collaborators:} Rebekah Cha, Jordan Powell

\medskip

\hrulefill

\begin{problems}
\section*{Part A}
\problem
\begin{problemparts}
\problempart $f_1(n) < f_2(n) < f_3(n) < f_5(n) < f_4(n) $
\problempart $f_4(n) < f_5(n) < f_3(n) < f_1(n) < f_2(n)$
\problempart $f_1(n) < f_2(n) < f_3(n) < f_5(n) \sim f_4(n) $
\end{problemparts}
\problem
\begin{problemparts}
\problempart $\theta (x)$
\problempart $\theta (x\log (x))$
\problempart $\theta (y\log (x))$
\problempart $\theta (\log x\log y)$
\problempart $\theta (x+y)$
\end{problemparts}
\problem
\problem
\problem
\problem

\section*{Part B}

\emph{Submit your implemented python script.}

\end{problems}

\end{document}
